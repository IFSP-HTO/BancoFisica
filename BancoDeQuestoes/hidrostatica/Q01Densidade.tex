
\begin{question}
Uma balança de braços iguais t\^em em um dos pratos um peso de 0.39 N 
e no outro prato um recipient e de peso desprezível. Sobre o recipiente existe 
uma torneira pingando 10 gotas de água por segundo, cada gota com
um volume de 4.7e-07 m³. Considerando a densidade da água 
1000 kg/m³ e g=10 m/s², determine o tempo necessário, 
em segundos, para que os pratos da balança fiquem nivelados. (OBS: apresente a resposta com uma casa decimal.)

\end{question}

\begin{solution}
%% Solution
Seja P o peso em um dos pratos, n o n\'umero de gotas, g a gravidade, $\rho$ a densidade da \'agua, V o volume da gota,  ent\~ao, depois que o peso do recipiente for tal que o sistema entre em equil\'ibrio, temos:
$P = n \times \rho \times V \times g $
Assim, isolando n temos:
$n =  \frac{P}{\times \rho \times V \times g} = \frac{0.39}{\times 1000 \times 4.7e-07 \times 10 = 82.9787234042553}$

O tempo necess\'ario para pingar esse n\'umero de gotas \'e dado por:

$T = \frac{n}{v_{gotas}} = 8.3$

Assim, depois de 8.3 segundos os braços da balança estarão equilibrados.
\end{solution}

%% META-INFORMATION
%% \extype{num}
%% \exsolution{8.3}
%% \exname{Q01Densidade}
%% \extol{0.1}
